\begin{enumerate}
    \item Y. H. Kim, S. Son, H.-C. Kim, B. Kim, Y.-G. Park, J. Nam, and J. Ryu, “Application
        of satellite remote sensing in monitoring dissolved oxygen variabilities: A case study for
        coastal waters in korea,” Environment International, vol. 134, p. 105301, 2020. [Online].
        Available: https://www.sciencedirect.com/science/article/pii/S0160412019327291
    \item NASA Ocean Biology Processing Group. (2018). MODIS-TERRA Level 3 Mapped Chlorophyll Data Version R2018.0 [Data set]. NASA Ocean Biology DAAC. https://doi.org/10.5067/TERRA/MODIS/L3M/CHL/2018
    \item R. F. Keeling, A. K¨ortzinger, and N. Gruber, “Ocean deoxygenation in a warming world,”
    Annual Review of Marine Science, vol. 2, no. 1, pp. 199-229, 2010, pMID: 21141663.
    [Online]. Available: https://doi.org/10.1146/annurev.marine.010908.163855
    \item K. Triana and A. J. Wahyudi, “Dissolved oxygen variability of indonesian seas over
    decades as detected by satellite remote sensing,” IOP Conference Series: Earth and
    Environmental Science, vol. 925, no. 1, p. 012003, nov 2021. [Online]. Available:
    https://doi.org/10.1088/1755-1315/925/1/012003
    \item M. H. Gholizadeh, A. M. Melesse, and L. Reddi, “A comprehensive review on water quality
    parameters estimation using remote sensing techniques,” Sensors (Switzerland), vol. 16, 8
    2016
    \item Chi, L., Song, X., Yuan, Y., Wang, W., Cao, X., Wu, Z., Yu, Z., 2020. Main factors 
    dominating the development, formation and dissipation of hypoxia off the 
    Changjiang Estuary (CE) and its adjacent waters, China. Environ. Pollut. https://doi. 
    org/10.1016/j.envpol.2020.
    \item Schulzweida, Uwe. (2021, October 31). CDO User Guide (Version 2.0.0). Zenodo. http://doi.org/10.5281/zenodo.5614769
\end{enumerate}